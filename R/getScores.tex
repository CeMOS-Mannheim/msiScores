\documentclass[]{article}
\usepackage{lmodern}
\usepackage{amssymb,amsmath}
\usepackage{ifxetex,ifluatex}
\usepackage{fixltx2e} % provides \textsubscript
\ifnum 0\ifxetex 1\fi\ifluatex 1\fi=0 % if pdftex
  \usepackage[T1]{fontenc}
  \usepackage[utf8]{inputenc}
\else % if luatex or xelatex
  \ifxetex
    \usepackage{mathspec}
  \else
    \usepackage{fontspec}
  \fi
  \defaultfontfeatures{Ligatures=TeX,Scale=MatchLowercase}
\fi
% use upquote if available, for straight quotes in verbatim environments
\IfFileExists{upquote.sty}{\usepackage{upquote}}{}
% use microtype if available
\IfFileExists{microtype.sty}{%
\usepackage{microtype}
\UseMicrotypeSet[protrusion]{basicmath} % disable protrusion for tt fonts
}{}
\usepackage[margin=1in]{geometry}
\usepackage{hyperref}
\hypersetup{unicode=true,
            pdftitle={getScores.R},
            pdfauthor={denis\_000},
            pdfborder={0 0 0},
            breaklinks=true}
\urlstyle{same}  % don't use monospace font for urls
\usepackage{color}
\usepackage{fancyvrb}
\newcommand{\VerbBar}{|}
\newcommand{\VERB}{\Verb[commandchars=\\\{\}]}
\DefineVerbatimEnvironment{Highlighting}{Verbatim}{commandchars=\\\{\}}
% Add ',fontsize=\small' for more characters per line
\usepackage{framed}
\definecolor{shadecolor}{RGB}{248,248,248}
\newenvironment{Shaded}{\begin{snugshade}}{\end{snugshade}}
\newcommand{\KeywordTok}[1]{\textcolor[rgb]{0.13,0.29,0.53}{\textbf{#1}}}
\newcommand{\DataTypeTok}[1]{\textcolor[rgb]{0.13,0.29,0.53}{#1}}
\newcommand{\DecValTok}[1]{\textcolor[rgb]{0.00,0.00,0.81}{#1}}
\newcommand{\BaseNTok}[1]{\textcolor[rgb]{0.00,0.00,0.81}{#1}}
\newcommand{\FloatTok}[1]{\textcolor[rgb]{0.00,0.00,0.81}{#1}}
\newcommand{\ConstantTok}[1]{\textcolor[rgb]{0.00,0.00,0.00}{#1}}
\newcommand{\CharTok}[1]{\textcolor[rgb]{0.31,0.60,0.02}{#1}}
\newcommand{\SpecialCharTok}[1]{\textcolor[rgb]{0.00,0.00,0.00}{#1}}
\newcommand{\StringTok}[1]{\textcolor[rgb]{0.31,0.60,0.02}{#1}}
\newcommand{\VerbatimStringTok}[1]{\textcolor[rgb]{0.31,0.60,0.02}{#1}}
\newcommand{\SpecialStringTok}[1]{\textcolor[rgb]{0.31,0.60,0.02}{#1}}
\newcommand{\ImportTok}[1]{#1}
\newcommand{\CommentTok}[1]{\textcolor[rgb]{0.56,0.35,0.01}{\textit{#1}}}
\newcommand{\DocumentationTok}[1]{\textcolor[rgb]{0.56,0.35,0.01}{\textbf{\textit{#1}}}}
\newcommand{\AnnotationTok}[1]{\textcolor[rgb]{0.56,0.35,0.01}{\textbf{\textit{#1}}}}
\newcommand{\CommentVarTok}[1]{\textcolor[rgb]{0.56,0.35,0.01}{\textbf{\textit{#1}}}}
\newcommand{\OtherTok}[1]{\textcolor[rgb]{0.56,0.35,0.01}{#1}}
\newcommand{\FunctionTok}[1]{\textcolor[rgb]{0.00,0.00,0.00}{#1}}
\newcommand{\VariableTok}[1]{\textcolor[rgb]{0.00,0.00,0.00}{#1}}
\newcommand{\ControlFlowTok}[1]{\textcolor[rgb]{0.13,0.29,0.53}{\textbf{#1}}}
\newcommand{\OperatorTok}[1]{\textcolor[rgb]{0.81,0.36,0.00}{\textbf{#1}}}
\newcommand{\BuiltInTok}[1]{#1}
\newcommand{\ExtensionTok}[1]{#1}
\newcommand{\PreprocessorTok}[1]{\textcolor[rgb]{0.56,0.35,0.01}{\textit{#1}}}
\newcommand{\AttributeTok}[1]{\textcolor[rgb]{0.77,0.63,0.00}{#1}}
\newcommand{\RegionMarkerTok}[1]{#1}
\newcommand{\InformationTok}[1]{\textcolor[rgb]{0.56,0.35,0.01}{\textbf{\textit{#1}}}}
\newcommand{\WarningTok}[1]{\textcolor[rgb]{0.56,0.35,0.01}{\textbf{\textit{#1}}}}
\newcommand{\AlertTok}[1]{\textcolor[rgb]{0.94,0.16,0.16}{#1}}
\newcommand{\ErrorTok}[1]{\textcolor[rgb]{0.64,0.00,0.00}{\textbf{#1}}}
\newcommand{\NormalTok}[1]{#1}
\usepackage{graphicx,grffile}
\makeatletter
\def\maxwidth{\ifdim\Gin@nat@width>\linewidth\linewidth\else\Gin@nat@width\fi}
\def\maxheight{\ifdim\Gin@nat@height>\textheight\textheight\else\Gin@nat@height\fi}
\makeatother
% Scale images if necessary, so that they will not overflow the page
% margins by default, and it is still possible to overwrite the defaults
% using explicit options in \includegraphics[width, height, ...]{}
\setkeys{Gin}{width=\maxwidth,height=\maxheight,keepaspectratio}
\IfFileExists{parskip.sty}{%
\usepackage{parskip}
}{% else
\setlength{\parindent}{0pt}
\setlength{\parskip}{6pt plus 2pt minus 1pt}
}
\setlength{\emergencystretch}{3em}  % prevent overfull lines
\providecommand{\tightlist}{%
  \setlength{\itemsep}{0pt}\setlength{\parskip}{0pt}}
\setcounter{secnumdepth}{0}
% Redefines (sub)paragraphs to behave more like sections
\ifx\paragraph\undefined\else
\let\oldparagraph\paragraph
\renewcommand{\paragraph}[1]{\oldparagraph{#1}\mbox{}}
\fi
\ifx\subparagraph\undefined\else
\let\oldsubparagraph\subparagraph
\renewcommand{\subparagraph}[1]{\oldsubparagraph{#1}\mbox{}}
\fi

%%% Use protect on footnotes to avoid problems with footnotes in titles
\let\rmarkdownfootnote\footnote%
\def\footnote{\protect\rmarkdownfootnote}

%%% Change title format to be more compact
\usepackage{titling}

% Create subtitle command for use in maketitle
\newcommand{\subtitle}[1]{
  \posttitle{
    \begin{center}\large#1\end{center}
    }
}

\setlength{\droptitle}{-2em}

  \title{getScores.R}
    \pretitle{\vspace{\droptitle}\centering\huge}
  \posttitle{\par}
    \author{denis\_000}
    \preauthor{\centering\large\emph}
  \postauthor{\par}
      \predate{\centering\large\emph}
  \postdate{\par}
    \date{Tue Aug 28 12:35:54 2018}


\begin{document}
\maketitle

Statistical \& multivariate scores for systematic workflow
standardization

Computes statistical \& multivariate scores to address the question of
reproducibility of sample preparation workflows. It is compatible with
mass spectrometry imaging as well as fingerprinting data (or any other
multivariate data). This function is the main function in the package
which calls internally the other functions available within this
package.

@param data.mat A matrix holding the intensity information of all
spectra with rows having the samples (pixels or spectra) and columns
denoting variables (m/z features in case of MS).

@param sample.labels A character vector of length
\code{nrow(data.matrix)} holding the labels of the samples (pixels or
spectra).

@param n.pc Number of principal components to be included into the
calculation of the scores. Defaults to 3. Note that the visulaization
will be always based on 3 PCs.

@param n.samples The size of the sampling pool for foldchange and
correlation analysis. Defaults to 3000.

@param compute.r2 A logical whether to compute coefficient of
determination (r2) via \code{randR2} method. Default is TRUE.

@param compute.fc A logical whether to perform natural fold change
analysis via \code{randR2} method. Default is TRUE.

@details This function tries to evaluate the relative similarity between
multivariate samples and can be used to assess reproducibility of data
generation for method development and standardization questions. It
generates a 3D plot of the supplied dataset \code{data.mat} color-coded
by the specific grouping provided by \code{sample.labels}. Along with
the plot, it generates unbiased scores that describe the relative
scatter and overlap of the samples supplied as a plot legend. These
scores include covariance per group, mean absolute deviation (MAD) per
group, euclidean distances between the centers of the groups, overall
within-class scatter, between-class scatter and a measure of group
overlap (J-overlap). Additionally, the function computes the coefficient
of determination based on randomized sub-sampling of the observations by
calling \code{randR2} generating a boxplot in the process. The function
also calls \code{natFC} function to evaluate the so called natural
foldchange on a paired randomly chosen combinations of observations
sampled from \code{data.mat}. For each pair of samples the function
computes \code{max(|5th FC precentile|, |95th FC percentile|)}. The
computed values are then pooled into a list named based on the supplied
groups as indicated in \code{sample.labels} and a boxplot is plotted.

Note that the within-class and between-class scatter as well as
covariance scores are generated by taking the trace of the matrices
rather than the determinants as this is more stable.

For more details about the generated scores and for citation please
refere to ``Erich, Katrin, et al. Biochimica et Biophysica Acta (BBA)-
Proteins and Proteomics 1865.7 (2017): 907-915''.

@return returns a list of four objects; the within-class scatter matrix
\code{Ws}, the between-class scatter matrix \code{Bs} , a list of
\code{R2} values per group and a list of \code{natFC} per group.
Additionally it generates four plots; the variance per principal
component, a boxplot of \code{R2} values per group, a boxplot of
\code{natFC} per group and an rgl 3D plot showing the PCA space with the
computed scores as a plot legend.

@export

@examples data(exampleScores) test \textless{}-
msiScores::getScores(data.mat = mat, sample.labels = lab)

@author Denis Abu Sammour, \email{d.abu-sammour@hs-mannheim.de}

@references Erich, Katrin, et al. ``Scores for standardization of
on-tissue digestion of formalin-fixed paraffin-embedded tissue in
MALDI-MS imaging.'' Biochimica et Biophysica Acta (BBA)-Proteins and
Proteomics 1865.7 (2017): 907-915.

\begin{Shaded}
\begin{Highlighting}[]
\NormalTok{getScores     =}\StringTok{ }\ControlFlowTok{function}\NormalTok{(data.mat, sample.labels, }\DataTypeTok{n.pc =} \DecValTok{3}\NormalTok{, }\DataTypeTok{n.samples =} \DecValTok{3000}\NormalTok{, }\DataTypeTok{compute.r2 =} \OtherTok{TRUE}\NormalTok{, }\DataTypeTok{compute.fc =} \OtherTok{TRUE}\NormalTok{)}
\NormalTok{\{}

       \CommentTok{#// define the return list}
\NormalTok{       returnList                  =}\StringTok{ }\KeywordTok{list}\NormalTok{()}

       \CommentTok{#// perfom pca}

       \KeywordTok{cat}\NormalTok{(}\StringTok{"performing PCA .. "}\NormalTok{, }\StringTok{"}\CharTok{\textbackslash{}n}\StringTok{"}\NormalTok{)}
\NormalTok{       pc                            =}\StringTok{ }\KeywordTok{prcomp}\NormalTok{(}\DataTypeTok{x =}\NormalTok{ data.mat, }\DataTypeTok{retx =} \OtherTok{TRUE}\NormalTok{, }\DataTypeTok{center =}\NormalTok{ T, }\DataTypeTok{scale =}\NormalTok{ F)}

       \CommentTok{#// plot the variances}
       \KeywordTok{windows}\NormalTok{()}
\NormalTok{       msiScores}\OperatorTok{::}\KeywordTok{pcChart}\NormalTok{(pc)}

\NormalTok{       coords                        =}\StringTok{ }\KeywordTok{as.data.frame}\NormalTok{(pc}\OperatorTok{$}\NormalTok{x[ , }\DecValTok{1}\OperatorTok{:}\NormalTok{n.pc]) }\CommentTok{# extract the points coordinates in pca}

       \CommentTok{#// RGL - plot 1st figure without ellipsoids}

       \CommentTok{#// find the center for each axis}
\NormalTok{       lim <-}\StringTok{ }\ControlFlowTok{function}\NormalTok{(x)\{}\KeywordTok{c}\NormalTok{(}\KeywordTok{min}\NormalTok{(x), }\KeywordTok{max}\NormalTok{(x)) }\OperatorTok{*}\StringTok{ }\FloatTok{1.1}\NormalTok{\}}
\NormalTok{       xlim                          =}\StringTok{ }\KeywordTok{lim}\NormalTok{(coords[,}\DecValTok{1}\NormalTok{])}
\NormalTok{       ylim                          =}\StringTok{ }\KeywordTok{lim}\NormalTok{(coords[,}\DecValTok{2}\NormalTok{])}
\NormalTok{       zlim                          =}\StringTok{ }\KeywordTok{lim}\NormalTok{(coords[,}\DecValTok{3}\NormalTok{])}

\NormalTok{       axes                          =}\StringTok{ }\KeywordTok{rbind}\NormalTok{(}\KeywordTok{c}\NormalTok{(}\KeywordTok{median}\NormalTok{(xlim), ylim[}\DecValTok{1}\NormalTok{] }\OperatorTok{*}\DecValTok{1}\NormalTok{, zlim[}\DecValTok{2}\NormalTok{]}\OperatorTok{*}\DecValTok{1}\NormalTok{),}
                                             \KeywordTok{c}\NormalTok{(xlim[}\DecValTok{2}\NormalTok{] }\OperatorTok{*}\StringTok{ }\DecValTok{1}\NormalTok{, }\KeywordTok{median}\NormalTok{(ylim), zlim[}\DecValTok{1}\NormalTok{] }\OperatorTok{*}\StringTok{ }\DecValTok{1}\NormalTok{),}
                                             \KeywordTok{c}\NormalTok{(xlim[}\DecValTok{2}\NormalTok{] }\OperatorTok{*}\StringTok{ }\DecValTok{1}\NormalTok{, ylim[}\DecValTok{1}\NormalTok{] }\OperatorTok{*}\StringTok{ }\DecValTok{1}\NormalTok{, }\KeywordTok{median}\NormalTok{(zlim)))}

       \CommentTok{#// assign points to  groups}
\NormalTok{       pts                           =}\StringTok{ }\KeywordTok{list}\NormalTok{()            }\CommentTok{# to hold the points coordinates in pca spcace}
\NormalTok{       ctrs                          =}\StringTok{ }\KeywordTok{list}\NormalTok{()            }\CommentTok{# to hold the centers of groups}
\NormalTok{       uniqueLabels                  =}\StringTok{ }\KeywordTok{unique}\NormalTok{(sample.labels)}
\NormalTok{       numGroups                     =}\StringTok{ }\KeywordTok{length}\NormalTok{(uniqueLabels)}

       \ControlFlowTok{for}\NormalTok{ (igroup }\ControlFlowTok{in} \DecValTok{1} \OperatorTok{:}\StringTok{ }\NormalTok{numGroups)}
\NormalTok{       \{}
\NormalTok{              pts[[igroup]]       =}\StringTok{ }\NormalTok{coords[}\KeywordTok{which}\NormalTok{(sample.labels }\OperatorTok{==}\StringTok{ }\NormalTok{uniqueLabels[igroup]), ]}
\NormalTok{              ctrs[[igroup]]      =}\StringTok{ }\KeywordTok{colMeans}\NormalTok{(pts[[igroup]])}
\NormalTok{       \}}

       \CommentTok{#// workout the colors}
       \CommentTok{#cdf                         = with(data.frame(labels = sample.labels),}
       \CommentTok{#                                   data.frame(labels = uniqueLabels,}
       \CommentTok{#                                              color = RColorBrewer::brewer.pal(n = length(uniqueLabels), name = "Set1")))}

\NormalTok{       tmpCol                      =}\StringTok{ }\KeywordTok{ifelse}\NormalTok{(}\KeywordTok{length}\NormalTok{(uniqueLabels) }\OperatorTok{<}\StringTok{ }\DecValTok{3}\NormalTok{, }\DecValTok{3}\NormalTok{, }\KeywordTok{length}\NormalTok{(uniqueLabels)) }\CommentTok{# used only for Ebrewer.pal function}

\NormalTok{       cdf                         =}\StringTok{ }\KeywordTok{merge}\NormalTok{(}\KeywordTok{data.frame}\NormalTok{(}\DataTypeTok{labels =}\NormalTok{ sample.labels, }\DataTypeTok{stringsAsFactors =}\NormalTok{ F),}
                                           \KeywordTok{data.frame}\NormalTok{(}\DataTypeTok{labels =}\NormalTok{ uniqueLabels,}
                                           \DataTypeTok{color =}\NormalTok{ RColorBrewer}\OperatorTok{::}\KeywordTok{brewer.pal}\NormalTok{(}\DataTypeTok{n =}\NormalTok{ tmpCol, }\DataTypeTok{name =} \StringTok{"Set1"}\NormalTok{)[}\DecValTok{1}\OperatorTok{:}\NormalTok{numGroups],}
                                           \DataTypeTok{stringsAsFactors =}\NormalTok{ F))}

\NormalTok{       cols                        =}\StringTok{ }\NormalTok{cdf}\OperatorTok{$}\NormalTok{color}
\NormalTok{       uniqueCols                  =}\StringTok{ }\KeywordTok{unique}\NormalTok{(cdf}\OperatorTok{$}\NormalTok{color)}

       \CommentTok{#// find the set of all combinations of all groups - this is needed to draw the distances between groups}
\NormalTok{       distCombs                     =}\StringTok{ }\KeywordTok{combn}\NormalTok{(}\KeywordTok{seq}\NormalTok{(}\DecValTok{1}\NormalTok{, numGroups), }\DataTypeTok{m =} \DecValTok{2}\NormalTok{)}

       \CommentTok{#// append the above combination matirx with corresponding euclidean distances between respective centroids}
\NormalTok{       distCombs                     =}\StringTok{ }\KeywordTok{data.frame}\NormalTok{(}\DataTypeTok{firstGroup =}\NormalTok{ distCombs[}\DecValTok{1}\NormalTok{, ],}
                                                  \DataTypeTok{secondGroup =}\NormalTok{ distCombs[}\DecValTok{2}\NormalTok{, ],}
                                                  \DataTypeTok{EuDistance =} \OtherTok{NA}\NormalTok{,}
                                                  \DataTypeTok{colour =} \KeywordTok{sort}\NormalTok{(}\KeywordTok{rainbow}\NormalTok{(}\KeywordTok{ncol}\NormalTok{(distCombs) }\OperatorTok{*}\StringTok{ }\DecValTok{3}\NormalTok{), }\DataTypeTok{decreasing =} \OtherTok{TRUE}\NormalTok{)[}\DecValTok{1}\OperatorTok{:}\KeywordTok{ncol}\NormalTok{(distCombs)],}

                                                  \DataTypeTok{stringsAsFactors =} \OtherTok{FALSE}\NormalTok{)}

\NormalTok{       withinGroupVariance           =}\StringTok{ }\KeywordTok{vector}\NormalTok{(}\StringTok{"numeric"}\NormalTok{, numGroups)}

       \CommentTok{#// To compute the within class scatter and between class scatter:}

\NormalTok{       Ws                            =}\StringTok{ }\NormalTok{DiscriMiner}\OperatorTok{::}\KeywordTok{withinSS}\NormalTok{(}\DataTypeTok{variables =}\NormalTok{ coords, }\DataTypeTok{group =}\NormalTok{ sample.labels)}
\NormalTok{       Bs                            =}\StringTok{ }\NormalTok{DiscriMiner}\OperatorTok{::}\KeywordTok{betweenSS}\NormalTok{(}\DataTypeTok{variables =}\NormalTok{ coords, }\DataTypeTok{group =}\NormalTok{ sample.labels)}

\NormalTok{       returnList}\OperatorTok{$}\NormalTok{Ws               =}\StringTok{ }\NormalTok{Ws}
\NormalTok{       returnList}\OperatorTok{$}\NormalTok{BS               =}\StringTok{ }\NormalTok{Bs}

       \CommentTok{#// MAD value}

\NormalTok{       MAD                         =}\StringTok{ }\KeywordTok{c}\NormalTok{()}

       \ControlFlowTok{for}\NormalTok{ (iii }\ControlFlowTok{in} \DecValTok{1} \OperatorTok{:}\StringTok{ }\KeywordTok{length}\NormalTok{(uniqueLabels))}
\NormalTok{       \{}
\NormalTok{              idx           =}\StringTok{ }\KeywordTok{which}\NormalTok{(sample.labels }\OperatorTok{==}\StringTok{ }\NormalTok{uniqueLabels[iii])}
\NormalTok{              numPoints     =}\StringTok{ }\KeywordTok{length}\NormalTok{(idx)}
\NormalTok{              coords_       =}\StringTok{ }\NormalTok{coords[idx, ]}
\NormalTok{              roiCenter     =}\StringTok{ }\KeywordTok{colMeans}\NormalTok{(coords_[ , ])}

\NormalTok{              distances     =}\StringTok{ }\NormalTok{(}\KeywordTok{rowSums}\NormalTok{(}\KeywordTok{abs}\NormalTok{(coords_ }\OperatorTok{-}\StringTok{ }\NormalTok{roiCenter))) }\CommentTok{#sqrt(rowSums((coords_ - roiCenter)**2))}

\NormalTok{              MAD[iii]      =}\StringTok{ }\KeywordTok{mean}\NormalTok{(distances)}
\NormalTok{       \}}

\NormalTok{       sdMAD                       =}\StringTok{ }\KeywordTok{sd}\NormalTok{(MAD, }\DataTypeTok{na.rm =} \OtherTok{TRUE}\NormalTok{)}

       \ControlFlowTok{for}\NormalTok{ (ictrs }\ControlFlowTok{in} \DecValTok{1} \OperatorTok{:}\StringTok{ }\KeywordTok{nrow}\NormalTok{(distCombs))}
\NormalTok{       \{}
\NormalTok{              distCombs}\OperatorTok{$}\NormalTok{EuDistance[ictrs]   =}\StringTok{ }\KeywordTok{sqrt}\NormalTok{(}\KeywordTok{sum}\NormalTok{((ctrs[[distCombs}\OperatorTok{$}\NormalTok{firstGroup[ictrs]]] }\OperatorTok{-}\StringTok{ }\NormalTok{ctrs[[distCombs}\OperatorTok{$}\NormalTok{secondGroup[ictrs]]])}\OperatorTok{**}\DecValTok{2}\NormalTok{))}
\NormalTok{       \}}

       \CommentTok{#// find the within-group variances}
       \ControlFlowTok{for}\NormalTok{ (icov }\ControlFlowTok{in} \DecValTok{1} \OperatorTok{:}\StringTok{ }\NormalTok{numGroups)}
\NormalTok{       \{}
\NormalTok{              covMat                                  =}\StringTok{ }\KeywordTok{cov}\NormalTok{(}\DataTypeTok{x =}\NormalTok{ coords[}\KeywordTok{which}\NormalTok{(sample.labels }\OperatorTok{==}\StringTok{ }\NormalTok{uniqueLabels[icov]), ])}
\NormalTok{              withinGroupVariance[icov]               =}\StringTok{ }\KeywordTok{sum}\NormalTok{(}\KeywordTok{diag}\NormalTok{(covMat)) }\CommentTok{# based on the trace not deteminant}

\NormalTok{       \}}

       \ControlFlowTok{if}\NormalTok{(compute.r2)}
\NormalTok{       \{}
              \CommentTok{#// compute the correlations per group}
              \KeywordTok{cat}\NormalTok{(}\StringTok{"computing coefficient of determination on randomly chosen subset of spectra per group .. "}\NormalTok{, }\StringTok{"}\CharTok{\textbackslash{}n}\StringTok{"}\NormalTok{)}
\NormalTok{              r2                                               =}\StringTok{ }\NormalTok{msiScores}\OperatorTok{::}\KeywordTok{randR2}\NormalTok{(}\DataTypeTok{data.mat =}\NormalTok{ data.mat, }\DataTypeTok{sample.labels =}\NormalTok{ sample.labels, }\DataTypeTok{n.samples =}\NormalTok{ n.samples)}

\NormalTok{              returnList}\OperatorTok{$}\NormalTok{R2List =}\StringTok{ }\NormalTok{r2}

\NormalTok{       \}}

       \ControlFlowTok{if}\NormalTok{(compute.fc)}
\NormalTok{       \{}
              \CommentTok{#// compute foldchange}
              \KeywordTok{cat}\NormalTok{(}\StringTok{"computing natural foldchange on randomly chosen subset of spectra per group .. "}\NormalTok{, }\StringTok{"}\CharTok{\textbackslash{}n}\StringTok{"}\NormalTok{)}
\NormalTok{              fc                                               =}\StringTok{ }\NormalTok{msiScores}\OperatorTok{::}\KeywordTok{natFC}\NormalTok{(}\DataTypeTok{data.mat =}\NormalTok{ data.mat, }\DataTypeTok{sample.labels =}\NormalTok{ sample.labels, }\DataTypeTok{n.samples =}\NormalTok{ n.samples)}

\NormalTok{              returnList}\OperatorTok{$}\NormalTok{FCList =}\StringTok{ }\NormalTok{fc}
\NormalTok{       \}}

       \KeywordTok{cat}\NormalTok{(}\StringTok{"plotting pca .. "}\NormalTok{, }\StringTok{"}\CharTok{\textbackslash{}n}\StringTok{"}\NormalTok{)}
\NormalTok{       rgl}\OperatorTok{::}\KeywordTok{open3d}\NormalTok{()}
\NormalTok{       rgl}\OperatorTok{::}\KeywordTok{par3d}\NormalTok{(}\DataTypeTok{windowRect =} \KeywordTok{c}\NormalTok{(}\DecValTok{0}\NormalTok{,}\DecValTok{23}\NormalTok{,}\DecValTok{1920}\NormalTok{,}\DecValTok{1040}\NormalTok{))}
\NormalTok{       rgl}\OperatorTok{::}\KeywordTok{rgl.viewpoint}\NormalTok{(}\DataTypeTok{theta =} \DecValTok{15}\NormalTok{, }\DataTypeTok{phi =} \DecValTok{3}\NormalTok{, }\DataTypeTok{fov =} \DecValTok{50}\NormalTok{, }\DataTypeTok{zoom =} \FloatTok{0.9}\NormalTok{)}
       \CommentTok{#par3d(cex = 2.5)}

\NormalTok{       rgl}\OperatorTok{::}\KeywordTok{par3d}\NormalTok{(}\DataTypeTok{cex =} \DecValTok{2}\NormalTok{)}
\NormalTok{       rgl}\OperatorTok{::}\KeywordTok{plot3d}\NormalTok{(coords[ , }\DecValTok{1}\OperatorTok{:}\DecValTok{3}\NormalTok{], }\DataTypeTok{xlab =} \StringTok{""}\NormalTok{, }\DataTypeTok{ylab =} \StringTok{""}\NormalTok{, }\DataTypeTok{zlab =} \StringTok{""}\NormalTok{, }\DataTypeTok{size =} \DecValTok{20}\NormalTok{, }\DataTypeTok{col =}\NormalTok{ cols, }\DataTypeTok{box =}\NormalTok{ F, }\DataTypeTok{alpha =} \FloatTok{0.5}\NormalTok{, }\DataTypeTok{axes =}\NormalTok{ F)       }\CommentTok{# plot the pixel points - All}

\NormalTok{       rgl}\OperatorTok{::}\KeywordTok{axes3d}\NormalTok{(}\DataTypeTok{edges =} \KeywordTok{c}\NormalTok{(}\StringTok{"x-+"}\NormalTok{, }\StringTok{"y+-"}\NormalTok{, }\StringTok{"z+-"}\NormalTok{), }\DataTypeTok{labels =} \OtherTok{TRUE}\NormalTok{, }\DataTypeTok{xlab =} \StringTok{"PC1"}\NormalTok{, }\DataTypeTok{ylab =} \StringTok{"PC2"}\NormalTok{, }\DataTypeTok{zlab =} \StringTok{"PC3"}\NormalTok{,}\DataTypeTok{tick =} \OtherTok{TRUE}\NormalTok{, }\DataTypeTok{nticks =} \DecValTok{5}\NormalTok{,}
                   \DataTypeTok{box=}\NormalTok{F, }\DataTypeTok{expand =} \FloatTok{1.03}\NormalTok{, }\DataTypeTok{lwd =} \DecValTok{4}\NormalTok{, }\DataTypeTok{scale =} \DecValTok{10}\NormalTok{, }\DataTypeTok{line =} \KeywordTok{c}\NormalTok{(}\OperatorTok{-}\DecValTok{60}\NormalTok{,}\DecValTok{0}\NormalTok{,}\DecValTok{0}\NormalTok{))}
\NormalTok{       rgl}\OperatorTok{::}\KeywordTok{grid3d}\NormalTok{(}\DataTypeTok{side =} \KeywordTok{c}\NormalTok{( }\StringTok{"y-+"}\NormalTok{, }\StringTok{"y++"}\NormalTok{))}
\NormalTok{       rgl}\OperatorTok{::}\KeywordTok{rgl.texts}\NormalTok{(axes[}\DecValTok{1}\NormalTok{, ], }\DataTypeTok{text =} \StringTok{"PC1"}\NormalTok{, }\DataTypeTok{adj =} \KeywordTok{c}\NormalTok{(}\DecValTok{2}\NormalTok{, }\FloatTok{2.8}\NormalTok{), }\DataTypeTok{color =} \StringTok{"black"}\NormalTok{, }\DataTypeTok{size =} \DecValTok{3}\NormalTok{)}
\NormalTok{       rgl}\OperatorTok{::}\KeywordTok{rgl.texts}\NormalTok{(axes[}\DecValTok{2}\NormalTok{, ], }\DataTypeTok{text =} \StringTok{"PC2"}\NormalTok{, }\DataTypeTok{adj =} \KeywordTok{c}\NormalTok{(}\OperatorTok{-}\FloatTok{1.5}\NormalTok{, }\DecValTok{0}\NormalTok{), }\DataTypeTok{color =} \StringTok{"black"}\NormalTok{, }\DataTypeTok{size =} \DecValTok{3}\NormalTok{)}
\NormalTok{       rgl}\OperatorTok{::}\KeywordTok{rgl.texts}\NormalTok{(axes[}\DecValTok{3}\NormalTok{, ], }\DataTypeTok{text =} \StringTok{"PC3"}\NormalTok{, }\DataTypeTok{adj =} \KeywordTok{c}\NormalTok{(}\OperatorTok{-}\FloatTok{1.5}\NormalTok{, }\FloatTok{2.5}\NormalTok{), }\DataTypeTok{color =} \StringTok{"black"}\NormalTok{, }\DataTypeTok{size =} \DecValTok{3}\NormalTok{)}
       \CommentTok{#rgl.bg(color = "beige")}

       \KeywordTok{palette}\NormalTok{(}\KeywordTok{as.character}\NormalTok{(uniqueCols)) }\CommentTok{# adjust the palette cause for some reason legend3d depends on it}
       \CommentTok{#palette(as.character(cdf$color[which(cdf$labels %in% uniqueLabels)]))}

\NormalTok{       rgl}\OperatorTok{::}\KeywordTok{legend3d}\NormalTok{(}\StringTok{"topleft"}\NormalTok{ , }\DataTypeTok{horiz =} \OtherTok{FALSE}\NormalTok{,}
                \DataTypeTok{title =} \StringTok{"PCA Space"}\NormalTok{,}
                \KeywordTok{c}\NormalTok{(uniqueLabels,}
                  \KeywordTok{paste}\NormalTok{(}\StringTok{"covariance = "}\NormalTok{, }\KeywordTok{round}\NormalTok{(withinGroupVariance, }\DataTypeTok{digits =} \DecValTok{2}\NormalTok{)),}
                  \KeywordTok{paste}\NormalTok{(}\StringTok{"MAD = "}\NormalTok{, }\KeywordTok{round}\NormalTok{(MAD, }\DataTypeTok{digits =} \DecValTok{2}\NormalTok{)),}
                  \KeywordTok{paste}\NormalTok{(}\StringTok{"Eu distance = "}\NormalTok{, }\KeywordTok{round}\NormalTok{(distCombs}\OperatorTok{$}\NormalTok{EuDistance, }\DataTypeTok{digits =} \DecValTok{2}\NormalTok{)),}
                  \KeywordTok{paste}\NormalTok{(}\StringTok{"Within-class var = "}\NormalTok{, }\KeywordTok{round}\NormalTok{(}\KeywordTok{sum}\NormalTok{(}\KeywordTok{diag}\NormalTok{(Ws)), }\DataTypeTok{digits =} \DecValTok{2}\NormalTok{)),}
                  \KeywordTok{paste}\NormalTok{(}\StringTok{"Between-class var = "}\NormalTok{, }\KeywordTok{round}\NormalTok{(}\KeywordTok{sum}\NormalTok{(}\KeywordTok{diag}\NormalTok{(Bs)), }\DataTypeTok{digits =} \DecValTok{2}\NormalTok{)),}
                  \KeywordTok{paste}\NormalTok{(}\StringTok{"J-overlap = "}\NormalTok{, }\KeywordTok{round}\NormalTok{(}\KeywordTok{sum}\NormalTok{(}\KeywordTok{diag}\NormalTok{(Ws))}\OperatorTok{/}\KeywordTok{sum}\NormalTok{(}\KeywordTok{diag}\NormalTok{(Bs)), }\DataTypeTok{digits =} \DecValTok{2}\NormalTok{))),}
                \DataTypeTok{pch =} \KeywordTok{c}\NormalTok{(}\KeywordTok{rep}\NormalTok{(}\DecValTok{16}\NormalTok{, numGroups),}
                        \KeywordTok{rep}\NormalTok{(}\DecValTok{9}\NormalTok{, numGroups),}
                        \KeywordTok{rep}\NormalTok{(}\DecValTok{8}\NormalTok{, numGroups),}
                        \KeywordTok{rep}\NormalTok{(}\DecValTok{18}\NormalTok{, }\KeywordTok{length}\NormalTok{(distCombs}\OperatorTok{$}\NormalTok{EuDistance)),}
                        \KeywordTok{rep}\NormalTok{(}\DecValTok{15}\NormalTok{, }\DecValTok{3}\NormalTok{)),}
                \DataTypeTok{col =} \KeywordTok{c}\NormalTok{(uniqueCols,}
\NormalTok{                        uniqueCols,}
\NormalTok{                        uniqueCols,}
\NormalTok{                        distCombs}\OperatorTok{$}\NormalTok{colour,}
                        \KeywordTok{rep}\NormalTok{(}\StringTok{"black"}\NormalTok{, }\DecValTok{3}\NormalTok{)),}
                \DataTypeTok{cex=}\FloatTok{2.5}\NormalTok{, }\DataTypeTok{inset=}\KeywordTok{c}\NormalTok{(}\FloatTok{0.02}\NormalTok{), }\DataTypeTok{bty =} \StringTok{"n"}\NormalTok{ ) }\CommentTok{#}

       \CommentTok{#// draw the ellipsoids}
       \ControlFlowTok{for}\NormalTok{ (i }\ControlFlowTok{in} \DecValTok{1}\OperatorTok{:}\NormalTok{numGroups) \{}

\NormalTok{              ellips =}\StringTok{ }\NormalTok{rgl}\OperatorTok{::}\KeywordTok{ellipse3d}\NormalTok{(}\KeywordTok{cov}\NormalTok{(pts[[i]][ , }\DecValTok{1}\OperatorTok{:}\DecValTok{3}\NormalTok{]), }\DataTypeTok{centre =}\NormalTok{ ctrs[[i]][}\DecValTok{1}\OperatorTok{:}\DecValTok{3}\NormalTok{], }\DataTypeTok{level =} \FloatTok{0.95}\NormalTok{)}

\NormalTok{              rgl}\OperatorTok{::}\KeywordTok{shade3d}\NormalTok{(ellips, }\DataTypeTok{col =}\NormalTok{ uniqueCols[i], }\DataTypeTok{alpha =} \FloatTok{0.1}\NormalTok{, }\DataTypeTok{lit =} \FloatTok{0.95}\NormalTok{)}

\NormalTok{              rgl}\OperatorTok{::}\KeywordTok{wire3d}\NormalTok{(ellips, }\DataTypeTok{col =} \StringTok{"black"}\NormalTok{, }\DataTypeTok{lit =} \OtherTok{FALSE}\NormalTok{, }\DataTypeTok{alpha =} \FloatTok{0.2}\NormalTok{)}

\NormalTok{              rgl}\OperatorTok{::}\KeywordTok{rgl.spheres}\NormalTok{(ctrs[[i]][}\DecValTok{1}\OperatorTok{:}\DecValTok{3}\NormalTok{], }\DataTypeTok{r =} \KeywordTok{diff}\NormalTok{(xlim) }\OperatorTok{/}\StringTok{ }\DecValTok{40}\NormalTok{, }\DataTypeTok{col =}\NormalTok{ uniqueCols[i] )}

              \CommentTok{#text3d(centers[[i]], texts = titles[i], cex = 2, adj = c(0,-1), col = "black" )}
              \CommentTok{#if (i == numGroups) \{distin =  ctrs[[1]]\} else \{distin = ctrs[[i + 1]]\}}

\NormalTok{       \}}

       \CommentTok{#// draw the distance lines}
       \ControlFlowTok{for}\NormalTok{ (i }\ControlFlowTok{in} \DecValTok{1} \OperatorTok{:}\StringTok{ }\KeywordTok{nrow}\NormalTok{(distCombs))}
\NormalTok{       \{}
\NormalTok{              rgl}\OperatorTok{::}\KeywordTok{lines3d}\NormalTok{(}\KeywordTok{rbind}\NormalTok{(ctrs[[distCombs}\OperatorTok{$}\NormalTok{firstGroup[i]]][}\DecValTok{1}\OperatorTok{:}\DecValTok{3}\NormalTok{], ctrs[[distCombs}\OperatorTok{$}\NormalTok{secondGroup[i]]][}\DecValTok{1}\OperatorTok{:}\DecValTok{3}\NormalTok{]),}
                      \DataTypeTok{lwd =} \DecValTok{5}\NormalTok{,  }\DataTypeTok{col =}\NormalTok{ distCombs}\OperatorTok{$}\NormalTok{colour[i], }\DataTypeTok{lty =} \StringTok{"dashed"}\NormalTok{, }\DataTypeTok{alpha =} \FloatTok{0.8}\NormalTok{)}
\NormalTok{       \}}

\NormalTok{       rgl}\OperatorTok{::}\KeywordTok{play3d}\NormalTok{(rgl}\OperatorTok{::}\KeywordTok{spin3d}\NormalTok{(}\DataTypeTok{axis =} \KeywordTok{c}\NormalTok{(}\DecValTok{0}\NormalTok{,}\DecValTok{1}\NormalTok{,}\DecValTok{0}\NormalTok{)), }\DataTypeTok{duration =} \DecValTok{12}\NormalTok{ )}

       \CommentTok{#rgl::snapshot3d(filename = "pca.png")}

       \KeywordTok{cat}\NormalTok{(}\StringTok{"}\CharTok{\textbackslash{}n}\StringTok{"}\NormalTok{)}

       \KeywordTok{cat}\NormalTok{(}\StringTok{"Done. }\CharTok{\textbackslash{}n}\StringTok{"}\NormalTok{)}

       \KeywordTok{return}\NormalTok{(returnList)}


\NormalTok{\}}
\end{Highlighting}
\end{Shaded}


\end{document}
